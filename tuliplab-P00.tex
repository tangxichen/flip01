%
% ---------------------------------------------------------------
% Copyright (C) 2012-2018 Gang Li
% ---------------------------------------------------------------
%
% This work is the default powerdot-tuliplab style test file and may be
% distributed and/or modified under the conditions of the LaTeX Project Public
% License, either version 1.3 of this license or (at your option) any later
% version. The latest version of this license is in
% http://www.latex-project.org/lppl.txt and version 1.3 or later is part of all
% distributions of LaTeX version 2003/12/01 or later.
%
% This work has the LPPL maintenance status "maintained".
%
% This Current Maintainer of this work is Gang Li.
%
%

\documentclass[
 size=12pt,
 paper=smartboard, %a4paper, smartboard, screen
 mode=present, %present, handout, print
 display=slides, % slidesnotes, notes, slides
% nohandoutpagebreaks,
% pauseslide,
style=tuliplab,
% nopagebreaks,clock
% hlentries=true,
% hlsections = true,
pauseslide,
fleqn,leqno]{powerdot}

\hypersetup{pdfpagemode=FullScreen}
% \usepackage[toc,highlight,blackslide,slidesonly,sounds,HA]{HA-prosper}
\usepackage{amssymb}
\usepackage{amsmath}
\usepackage{rotating}
\usepackage{graphicx}
\usepackage{boxedminipage}
\usepackage{media9}
\usepackage{rotate}
\usepackage{calc}
\usepackage[absolute]{textpos}
\usepackage{psfrag,overpic}
\usepackage{fouriernc}
\usepackage{pstricks,pst-node,pst-text,pst-3d,pst-grad}
\usepackage{moreverb,epsfig,color,subfigure}
\usepackage{color}
\usepackage{pstricks}
\usepackage{pstricks-add}
\usepackage{pst-text}
\usepackage{pst-node, pst-tree}
\usepackage{booktabs}
\usepackage{etex}
\usepackage{breqn}
\usepackage{multirow}
\usepackage{gitinfo2}


\usepackage{listings}
\lstset{frameround=fttt,
frame=trBL,
stringstyle=\ttfamily,
backgroundcolor=\color{yellow!20},
basicstyle=\footnotesize\ttfamily}
\lstnewenvironment{code}{
\lstset{frame=single,escapeinside=`',
backgroundcolor=\color{yellow!20},
basicstyle=\footnotesize\ttfamily}
}{}


\usepackage{fouriernc}
\usepackage{hyperref}

%%%%%%%%%%%%%%%%%%%%%%%%%%%%%%%%%%%%%%%%%%%%%%%%%%%%%%%%%%%%%%%%%%%%%%%%
% title
% TODO: Customize to your Own Title, Name, Address
%
\title{Identifying Customers
}
\author{
Xichen Tang
\\
QUT
% \href{mailto:gangli@acm.org}{gangli@acm.org}
% \and % more authors
}
\date{\today}



% Customize the setting of slides
\pdsetup{
% theslide=\arabic{slide}~/~\pageref*{lastslide},
% theslide=\arabic{slide},
rf=\href{http://www.tulip.org.au/members}{
Last Changed by: \textsc{Xichen Tang}\  (\today)
},
cf=\hyperlink{blankslide}{Identifying Customers},
trans=Fade,
list={labelsep=1em,leftmargin=*,itemsep=0pt,topsep=5pt,parsep=0pt},
% counters={theorem,lemma},
% randomdots,dmaxdots=80
}


\begin{document}

\maketitle
\section{Directory}
\begin{slide}[toc=,bm=]{Introduce}
\tableofcontents[content=sections,type=1]
\end{slide}
\section{Subject Introduce}
\begin{slide}[toc=,bm=]{Introduce}
\begin{itemize}
\item The kaggle subject:Santander Customer Transaction Prediction

\end{itemize}
In this challenge, we need to identify which customers will make a specific transaction in the future, irrespective of the amount of money transacted.
\includegraphics[width=6in]{face}
\end{slide}
\begin{slide}{Data}
\begin{itemize}
    \item train_data
\end{itemize}

\begin{tabular}{|c|c|c|c|c|c|c|}%一个c表示有一列,格式为居中显示(center)

ID_code&target&var_0&var_1&...&var_198&var_199\\%第一行第一列和第二列  中间用&连接
 \hline
train_0&0&8.9255 &-6.7863 &...  & 12.7803 & -1.0914  \\
 \hline
train_1&0&11.5006& -4.1473 &...  &18.3560  &   1.9518\\
\end{tabular}
\begin{itemize}
\item test.csv
\end{itemize}

\begin{tabular}{|c|c|c|c|c|c|}%一个c表示有一列,格式为居中显示(center)

ID_code&var_0&var_1&...&var_198&var_199\\%第一行第一列和第二列  中间用&连接
 \hline
test_0&8.9255 &-6.7863 &...  & 12.7803 & -1.0914  \\
 \hline
test_1&11.5006& -4.1473 &...  &18.3560  &   1.9518\\
\end{tabular}
\begin{itemize}
\item train_data.info
\end{itemize}

\begin{tabular}{|c|c|c|}%一个c表示有一列,格式为居中显示(center)
 \hline
RangeIndex:&200000 entries&0 to 199999  \\
 \hline
Columns:&202 entries&ID_code to var_199\\
\end{tabular}

\end{slide}

\begin{slide}{Avg and Std }
\begin{itemize}
\item by describe()
\par
\includegraphics[width=7in]{p011}

\end{itemize}
\end{slide}

\section{Naive Bayes}


\begin{slide}{Statistical Functions}
\begin{itemize}
\item Calculate Prob
\par
P(A|B)=$\frac{P(AB)}{P(B)}$
\par
\includegraphics[width=4in]{p012}
\item Smoothing
\par
If the probability value to be estimated is 0, the calculation result of posterior probability will be affected. The solution to this problem is to use smoothing
\par
%\includegraphics[width=4in]{p1}

\end{itemize}
\end{slide}


\begin{slide}{AUC }
\begin{itemize}
\item Validation AUC
\par
\includegraphics[width=4in]{AUC}
\par
Validation AUC = 0.905571412599524
\par
%\includegraphics[width=4in]{p2}
\end{itemize}
\end{slide}


\begin{slide}{Result}
\begin{enumerate}[type=1]%[label=\romani*)]
\item Probability
\par
\includegraphics[width=6in]{p013}
\end{enumerate}
\end{slide}



\begin{slide}{Decomposition of the graphics}
\begin{itemize}[type=1]
\item Result
\par
We can get Raw data, trend data, periodic data, random variables\par
%\includegraphics[width=4in]{p3}
\end{itemize}
\end{slide}





\begin{slide}{Test for stationarity}
\begin{itemize}
\item The Measure
\par
seasonal_decompose
\item The Result
\par
%\includegraphics[width=4in]{p5}
\par
There is a 15\% probability that the sequence is non-stationary
\end{itemize}
\end{slide}


\begin{slide}{Remove  Seasonalization }
\begin{itemize}[type=1]
\item remove seasonalization
\par
%\includegraphics[width=4in]{p9}
\par
\end{itemize}
\end{slide}



\begin{slide}{Remove  Seasonalization }
\begin{itemize}[type=1]
\item Result
\par
%\includegraphics[width=4in]{p7}
\par
The p value is very small, and the sequence after the difference is considered stable \par
Now after the transformations, our p-value for the DF test is well within 5 \%. Hence we can assume Stationarity of the series
\end{itemize}
\end{slide}





\section{Forecasts}
\begin{slide}{Modle }
\begin{itemize}[type=1]
\item modle
\par
%\includegraphics[width=4in]{p8}
\item result
\par
sm.tsa.arma_order_select_ic\par
get the best p and q values (time-consuming) by passing in the qualified maximum,It takes too long
\item get result
\par
Input the start time and end time for data prediction
then Restore the predicted value
\end{itemize}
\end{slide}

\section{Thanks and Question}

\end{document}

\endinput
