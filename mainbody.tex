%=================================================================
\section{Introduce} \label{sec-preliminaries}
At Santander our mission is to help people and businesses prosper. We are always looking for ways to help our customers understand their financial health and identify which products and services might help them achieve their monetary goals.

Our data science team is continually challenging our machine learning algorithms, working with the global data science community to make sure we can more accurately identify new ways to solve our most common challenge, binary classification problems such as: is a customer satisfied? Will a customer buy this product? Can a customer pay this loan?

\subsection{Data }
\

The data mainly includes the customer's label number, whether there will be specific transactions in the future, 1 and 0 represent yes or no respectively, and then various relevant information provided by the bank, without specific name.
\begin{description}
\item ID_code
\subitem customer ID
\item target
\subitem Whether it will trade in the future, 0 means no, 1 means yes
\item var_0
\subitem Relevant information provided by the bank
\item var_1
\subitem Relevant information provided by the bank
\item ...
\subitem Relevant information provided by the bank
\item var_199
\subitem Relevant information provided by the bank

\end{description}




\subsection{Train Data and Test Data }
\begin{itemize}
    \item train_data
\end{itemize}

\begin{tabular}{|c|c|c|c|c|c|c|}%一个c表示有一列,格式为居中显示(center)

ID_code&target&var_0&var_1&...&var_198&var_199\\%第一行第一列和第二列  中间用&连接
 \hline
train_0&0&8.9255 &-6.7863 &...  & 12.7803 & -1.0914  \\
 \hline
train_1&0&11.5006& -4.1473 &...  &18.3560  &   1.9518\\
\end{tabular}
\begin{itemize}
\item test.csv
\end{itemize}

\begin{tabular}{|c|c|c|c|c|c|}%一个c表示有一列,格式为居中显示(center)

ID_code&var_0&var_1&...&var_198&var_199\\%第一行第一列和第二列  中间用&连接
 \hline
test_0&8.9255 &-6.7863 &...  & 12.7803 & -1.0914  \\
 \hline
test_1&11.5006& -4.1473 &...  &18.3560  &   1.9518\\
\end{tabular}
\begin{itemize}
\item Missing Values
\par
train data missing values? False
\par
test data missing values?False
\par
There are no null values in our dataset. This is good thing else we need to handle the missing values.
\end{itemize}


\begin{itemize}
\item Calculate Avg and Std by describe()
\par
\includegraphics[width=5in]{p011}

\end{itemize}
\newpage
\section{Method}
We use different methods to predict, and finally through comparison, we get the best value
\begin{itemize}
\item Naive Bayes
\par

\item Gaussian naive Bayes
\par

\item LinearRegression
\par


\item Catboost
\par


\end{itemize}
\subsection{Naive Bayes}
\begin{itemize}
\item Calculate Prob
\par
P(A|B)=$\frac{P(AB)}{P(B)}$
\par
\includegraphics[width=4in]{p012}
\item Smoothing
\par
If the probability value to be estimated is 0, the calculation result of posterior probability will be affected. The solution to this problem is to use smoothing
\par
%\includegraphics[width=4in]{p1}
\item Validation AUC
\par
\includegraphics[width=4in]{AUC}
\par
Validation AUC = 0.905571412599524
\par
\item Probability
\par
\includegraphics[width=5in]{p013}
\end{itemize}


\subsection{Gaussian naive Bayes}
\begin{itemize}
\item Calculation of prior probability
\par
use Counter() maybe more convenient
\par
\item Avg and Std
\par
\item Calculate likelihood
\par
Using probability density function of Gaussian distribution to calculate likelihood and then multiply to get likelihood
We can get Raw data, trend data, periodic data, random variables
\par
\item Training model and get prediction
\par
The probabilities of each label are multiplied by the likelihood and then normalized to get the prob of each label.
\item AUC
\par
Validation AUC is 0.8051607443604657.
%\includegraphics[width=4in]{p3}
\end{itemize}




\subsection{LinearRegression}
\begin{itemize}
\item Merge test/train datasets
\par
\item Add more features
\par
Normalize the data,Standardization of normal distribution,then Square the value, cubic the value,Cumulative normal percentile,Normalize the data,again.Do linear regression,Write submission file
%\includegraphics[width=4in]{p5}
\par
AUC:  0.8025517936065763
\end{itemize}

\subsection{Catboost}
\begin{itemize}
\item Feature Correlations
\par
\includegraphics[width=5in]{feature}
\item Get the features
Get the top 100 features,merge them and divide the training set and test set
\item process data
\par
In catboost, you don't have to worry about this at all. You just need to tell the algorithm which features belong to category features, and it will help you deal with them automatically
\par
Finally, we feed the data to the algorithm and train it
\par
\includegraphics[width=5in]{cat}
\par
\item fit and prediction
\par
AUC: 0.80399151
\end{itemize}
\newpage
\section{Conclusion}
\begin{itemize}
\item Naive Bayes
\par
AUC: 0.9055714
\item Gaussian naive Bayes
\par
AUC: 0.8051607
\item LinearRegression
\par
AUC: 0.8025517

\item Catboost
\par
AUC: 0.8039915

\end{itemize} 